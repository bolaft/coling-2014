%
% File coling2014.tex
%
% Contact: jwagner@computing.dcu.ie
%%
%% Based on the style files for ACL-2014, which were, in turn,
%% Based on the style files for ACL-2013, which were, in turn,
%% Based on the style files for ACL-2012, which were, in turn,
%% based on the style files for ACL-2011, which were, in turn, 
%% based on the style files for ACL-2010, which were, in turn, 
%% based on the style files for ACL-IJCNLP-2009, which were, in turn,
%% based on the style files for EACL-2009 and IJCNLP-2008...

%% Based on the style files for EACL 2006 by 
%% e.agirre@ehu.es or Sergi.Balari@uab.es
%% and that of ACL 08 by Joakim Nivre and Noah Smith

\documentclass[11pt]{article}

\usepackage{coling2014}
\usepackage{times}
\usepackage{url}
\usepackage{hyperref}
\usepackage{breakurl}
\usepackage[utf8]{inputenc}
\usepackage[T1]{fontenc} 
\usepackage{latexsym}
\usepackage{todonotes}
\usepackage{csquotes}

\hypersetup{
    colorlinks=false,
    pdfborder={0 0 0},
}

%\setlength\titlebox{5cm}

% You can expand the titlebox if you need extra space
% to show all the authors. Please do not make the titlebox
% smaller than 5cm (the original size); we will check this
% in the camera-ready version and ask you to change it back.

\title{Act of speech boundary detection in email corpora}

\date{}

\begin{document}

\maketitle

\begin{abstract}

\end{abstract}

\section{Introduction}
\label{intro}

%
% The following footnote without marker is needed for the camera-ready
% version of the paper.
% Comment out the instructions (first text) and uncomment the 8 lines
% under final paper for your variant of English.
% 
\blfootnote{
    %
    % for review submission
    %
    \hspace{-0.65cm}  % space normally used by the marker
    % Place licence statement here for the camera-ready version, see
    % Section~\ref{licence} of the instructions for preparing a
    % manuscript.
    %
    % % final paper: en-uk version (to license, a licence)
    %
    % \hspace{-0.65cm}  % space normally used by the marker
    % This work is licensed under a Creative Commons 
    % Attribution 4.0 International Licence.
    % Page numbers and proceedings footer are added by
    % the organisers.
    % Licence details:
    % \url{http://creativecommons.org/licenses/by/4.0/}
    % 
    % % final paper: en-us version (to licence, a license)
    %
    % \hspace{-0.65cm}  % space normally used by the marker
    % This work is licenced under a Creative Commons 
    % Attribution 4.0 International License.
    % Page numbers and proceedings footer are added by
    % the organizers.
    % License details:
    % \url{http://creativecommons.org/licenses/by/4.0/}
}

\section{Related Work}

\section{Generating the annotations}

\section{Building the segmenter}

\subsection{Sequence labeling}

Our email speech act segmenter system is built around a linear-chain Conditional Random Field (CRF) classifier, as implemented in the sequence labelling toolkit Wapiti \cite{lavergne2010practical}. Each email is processed as a sequence and each sentence as an element of that sequence.

\subsection{Features for boundary detection}

The classifier uses features that capture graphic, orthographic, lexical and syntactic information about the sentence. Many of them are borrowed from related work in speech act classification \cite{qadir2011classifying} and email segmentation \cite{lampert2009segmenting}.

\subsubsection{Lexical and syntactic features}

\textbf{Ngrams:} we focus on the first and last three significant tokens in the sentence. Significance is determined by the number of occurrences of the token in the training corpus: terms with a frequency lower than 1/X \todo{fix me} are considered insignificant. If a sentence contains less than six tokens, the same token can be found in both triplets. For example, in the sentence \textit{``Have a good day !''}, the first three tokens would be \textit{``Have''}, \textit{``a''}, \textit{``good''} and the last three would be \textit{``good''}, \textit{``day''} and \textit{``!''}. If the sentence contains less than three tokens, missing values are replaced by a placeholder \textit{``null''} token.

We define three individual features for the three unigrams, the two bigrams and the single trigram found in each of these sets. The features are the following: the unaltered form of the token (case-sensitive), the lemmatized form of the token (case-insensitive) and the corresponding part-of-speech.

We use the Stanford Log-linear Part-Of-Speech Tagger for morpho-syntactic tagging \cite{toutanova2003feature}, and the WordNet lexical databse to perform lemmatization \cite{miller1995wordnet}.

\textbf{Question words and interrogative ngrams:} one feature checks if a sentence begins with a ``wh*'' question word  (\textit{``who''}, \textit{``when''}, \textit{``where''}, \textit{``what''}, \textit{``which''}, \textit{``what''}, \textit{``how''}) or an ngram suggesting an incoming interrogation (e.g. \textit{``is it''} or \textit{``are there''}), and another one checks if the sentence merely contains such a word or ngram.

\textbf{Modals:} one feature indicates wether the sentence contains a modal (\textit{``may''}, \textit{``must''}, \textit{``shall''}, \textit{``will''}, \textit{``might''}, \textit{``should''}, \textit{``would''}, \textit{``could''}, and their negative forms).

\textbf{Plan phrases:} one feature looks for plan phrases (e.g. \textit{``i will''} or \textit{``we are going to''})

\textbf{Personal words:} three features check for first person, second person and third persons words, respectively (e.g. \textit{``we''}, \textit{``my''} and \textit{``me''} are recognized as first person words).

\textbf{First personal pronoun:} one feature records the first personal pronoun found in the sentence.

\textbf{First verbal form:} one feature records the tag of the first verbal form found in the sentence as classified by the Stanford Part-Of-Speech tagger ; that is an element of the Penn Treebank tag set \footnote{Alphabetical list of part-of-speech tags used in the Penn Treebank Project: \url{http://www.ling.upenn.edu/courses/Fall_2003/ling001/penn_treebank_pos.html}} (e.g. the feature \textit{``VBZ''} indicates a present tense verb in third person singular).

\subsubsection{Graphic features}

The following features' values are relative to other sentences. Four classes are defined for each feature: ``highest'' (top 25\%), ``high'' (top 50\%), ``low'' (bottom 50\%) and ``lowest'' (bottom 25\%).

\textbf{Number of tokens:} the total number of tokens in the sentence, including insignificant ones.

\textbf{Number of characters:} the total number of characters in the sentence, including those in insignificant tokens.

\textbf{Average token length:} the average length of a token, including insignificant ones.

\textbf{Proportion of uppercase characters:} the proportion of uppercase characters in the sentence.

\textbf{Proportion of alphabetic characters:} the proportion of alphabetic characters in the sentence.

\textbf{Proportion of numeric characters:} the proportion of numeric characters in the sentence.

\subsubsection{Orthographic features}

\textbf{Number of greater-than signs:} the number of greater-than signs (``>''), also know as ``chevron'' symbols, in the sentence. Like previous features, this one is computed relatively to other sentences in the training set.

\textbf{Position:} the position of the sentence in the email.

\textbf{Question mark:} one feature checks if the sentence ends with a question mark, and another checks if it at least contains one.

\textbf{Colon:} one feature checks if the sentence ends with a colon, and another checks if it at least contains one.

\textbf{Semicolon:} one feature checks if the sentence ends with a semicolon, and another checks if it at least contains one.

\textbf{Early punctuation:} the last feature checks if the sentence contains any punctuation within the first three tokens of the sentence. This is meant to recognize greetings \cite{qadir2011classifying}.

\section{Experimental framework}

\subsection{Corpora}

\subsubsection{Ubuntu technical support email archive}

We use the ubuntu-users email archive\footnote{\url{https://lists.ubuntu.com/archives/ubuntu-users/}} as our primary corpus. It offers a number of advantages:

\begin{itemize}
    \item It is free, and distributed under an unrestrictive licence
    \item It is up-to-date, with messages as recent as late 2013, and therefore is representative of modern emailing
    \item A number of alternatives archives are available\footnote{\url{https://lists.ubuntu.com/archives/}}
    \item Archives are available in a number of languages, including some very resource-poor languages
\end{itemize}

\subsubsection{BC3 email corpus}

\subsection{Evaluation protocols}

\subsubsection{Metrics}

\subsubsection{Baselines}

\section{Experiments}

\subsection{Email segmentation}

\subsection{Contributions of the segmenter to a speech act classification task}

\subsection{Results and discussion}

\section{Conclusion}

% include your own bib file like this:
\bibliographystyle{acl}
\bibliography{refs}

\end{document}
