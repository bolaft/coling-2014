%
% File coling2014.tex
%
% Contact: jwagner@computing.dcu.ie
%%
%% Based on the style files for ACL-2014, which were, in turn,
%% Based on the style files for ACL-2013, which were, in turn,
%% Based on the style files for ACL-2012, which were, in turn,
%% based on the style files for ACL-2011, which were, in turn, 
%% based on the style files for ACL-2010, which were, in turn, 
%% based on the style files for ACL-IJCNLP-2009, which were, in turn,
%% based on the style files for EACL-2009 and IJCNLP-2008...

%% Based on the style files for EACL 2006 by 
%% e.agirre@ehu.es or Sergi.Balari@uab.es
%% and that of ACL 08 by Joakim Nivre and Noah Smith

\documentclass[11pt]{article}

\usepackage{coling2014}
\usepackage{times}
\usepackage{url}
\usepackage{hyperref}
\usepackage[utf8]{inputenc}
\usepackage[T1]{fontenc} 
\usepackage{latexsym}
\usepackage{upgreek}
\usepackage{bm}
\usepackage{todonotes}
\usepackage{csquotes}

\usepackage{setspace}
\usepackage{listings}

\usepackage{rst}
\usepackage{multicol}

\hypersetup{
    colorlinks=false,
    pdfborder={0 0 0},
}

%\setlength\titlebox{5cm}

% You can expand the titlebox if you need extra space
% to show all the authors. Please do not make the titlebox
% smaller than 5cm (the original size); we will check this
% in the camera-ready version and ask you to change it back.

\title{``Lazy'' manual annotation for training online discussion messages segmenters} %Act of speech boundary detection in email corpora

\date{}

\author{}

\begin{document}

\maketitle

\begin{abstract}
In this context of online discussion analysis, we propose an innovative manual strategies for annotation of dialog act segments.
\end{abstract}

% -----------------------------------------------------------------------------
% INTRODUCTION

%------------------------------------------------------------------------------
\section{Introduction}
\label{sec:intro}

Automatic processing of online conversations (forum, emails) 
% about information or assistance needs 
is a highly important issue for the industrial and the scientific communities which care to improve the present question/answering systems, identify emotions or intentions in customer requests or reviews, detect messages containing requests for action or unsolved severe problems\ldots

%L'analyse automatique des conversations écrites en ligne asynchrones (e.g. forum, courriel) portant des demandes d'information (e.g. dépannage et assistance) constituent un fort enjeu scientifique et industriel : amélioration des systèmes de question/réponse actuels, détection de revues négatives de produits par des clients, identification de problèmes sévères non-résolus... 



%2010_gupta_W10-0202_email-Emotion-Detection-in-Email-Customer-Care
%2012_vinodkumar_email_Annotations-for-Power-Relations-on-Email-Threads
%2010_kim-W10-0507_modeling-social-and-content-dynamics-in-discussion-forum
%2012_qu_E12-1037_recommendation-prediction-of-user-interest-in-forum
%2013_chen_N13-1124_Identifying-Intention-Posts-in-Discussion-Forums
%2011_marin_W11-0706_detecting-forum-authority-claims
%2010_wang_P10-1027_recommendation-in-forum-blog
%2009_Stavrianou_asonam_Definition and Measures of an Opinion Model for Mining Forums
%2009_arora_N09-2010_Identifying Types of Claims in Online Customer Reviews.pdf


%2013_ott_Negative Deceptive Opinion Spam

%2011_qu_I11-1164_finding-problem-solving-threads-in-forum
%2010_lampert_naacl_Detecting Emails Containing Requests for Action


%Les travaux existants se fondent généralement sur une pré-annotation des messages en actes du dialogue (\textit{DA}). Les \textit{DA} décrivent les fonctions communicatives portées dans chaque message (e.g. question, réponse, remerciement...) Austin FIXME. Kim FIXME proposent une liste d'actes spécifiques à la description des messages au sein de forums.
%Jusqu'à présent la majorité des travaux ont modélisé les discussions au niveau de leurs messages en les décrivant su

In most works, conversation interactions between the participants are modelled in terms of dialogue acts (DA) \cite{austin:1970}. The DAs describe the communicative function conveyed by each text utterance  (e.g.~question, answer, greeting,\ldots).
%La majorité des travaux modélisent les interactions entre intervenants en termes d'\textit{actes de dialogue} (DA) \cite{austin:1970}. Les DAs correspondent aux fonctions communicatives portées par chaque énoncé d'un texte (e.g. question, réponse, remerciement...). 
% La théorie des actes du dialogue \cite{austin:1970} propose de décrire les énoncés en termes des fonctions communicatives portées par chacun d'eux (e.g. question, réponse, remerciement...). 
%
The main trend in automatic DA recognition consists in using supervised learning algorithms to predict the DA conveyed by a sentence or a message \cite{joty:2013:sigdial}.
% L'approche dominante en reconnaissance automatique de DA consiste en l'usage d'algorithmes de classification supervisée \cite{joty:2013:sigdial} pour déterminer l'acte porté par une phrase ou un message. 
%
The hypothesized message segmentation results from the global analysis of these individual predictions over each sentence.
%Le découpage des messages découle alors des résultats d'analyse.

A first remark on this paradigm is that it is not realistic to use in the context of multi-domain and multimodal processing because it requires the building of training data which is a very substantial and time-consuming task.
%Une première critique de ce paradigme est que sa mise en oeuvre est laborieuse et coûteuse en temps d'annotation pour construire des données d'entraînement, ce qui n'est pas réaliste en contexte de traitement multi-domaine voire multi-modal.
%
A second remark is that the model does not have a fine-grained representation of the message structure or the relations between messages. Considering such characteristics could drastically improve the systems, for example to allow focus on specific text parts or to filter out the less relevant ones. 
% Une seconde critique est que dans ce contexte, la connaissance de la structure des messages permettrait aux applications susvisées de se focaliser sur des parties spécifiques et filtrer les passages moins pertinents. 
Indeed, apart from the closing formula, a message may for example be made of several distinct information requests, the description of unsuccessful procedure, the quote of third-party messages\ldots
% En effet, outre des formules de politesse, un message peut compter par exemple plusieurs expressions de besoin, décrire une procédure infructueuse et citer des portions de messages tiers. 
% faire référence à plusieurs contenus exprimés par différentes contributions tierces

So far, few works address the problem of email segmentation.
\cite{lampert:2009:emnlp} propose to segment emails in prototypical zones such as the author's contribution, the quotes of original messages, the signature, the opening and the closing formulas. 
In comparison, in this work we focus on the segmentation of the author's contribution (what we call the new content part in the message).
\cite{joty:2013:jair} identifies topical segments. In our approach we are more interested by rhetorically motivated segments.
%De par sa robustesse, cette dernière approche peut servir de base de comparaison.
%Or, peu de travaux se sont penchés sur la tâche de segmentation des courriels en tant que telle. Les rares travaux ne concernent pas le découpage en DA. \cite{lampert:2009:emnlp} s'intéressent à découper les courriels en des zones prototypiques telles que la contribution de l'auteur, les reprises de messages tiers, la signature et les formules d'appel et de clôture... \cite{joty:2013:jair} identifient des segments thématiques. De par sa robustesse, cette dernière approche peut servir de base de comparaison.

This paper addresses the problem of rhetorically segmenting the new content parts of messages in online discussions. The process aims at supporting the analysis of messages in terms of dialogue acts.
%Ce sujet propose de travailler sur la tâche de segmentation des courriels en français dans des discussions portant sur des demandes d'information. Ce travail vise à soutenir une analyse automatique des messages en DA.

Despite the drawbacks mentioned above, a supervised approach remains generally the most efficient and reliable method to solve classification problems in Natural Language Processing. 
% 
Our objective is to train a system to detect the segment boundaries; in other words to classify if a given sentence starts, ends or continues a segment.


\begin{figure}
\begin{minipage}{.63\textwidth}

%\begin{multicols}{2}[]

%\begin{multicols}{1}[]
    %    \centering
\fbox {
    \parbox{\linewidth}{
        \begin{subfigure}[b]{0.9\textwidth}
\small
%\footnotesize
  [Hi!]$^{S1}$\vspace{0.1cm}

[I got my ubuntu cds today and i'm really impressed.]$^{S2}$ [My\\ \ 
friends like them and my teachers too (i'm a student).]$^{S3}$ \\ \ 
[It's really funny to see, how people like ubuntu and start feeling geek\\ \ 
and blaming microsoft when they use it.]$^{S4}$\vspace{0.1cm}

[Unfortunately everyone wants an ubuntu cd, so can i download the cd\\ \ 
covers anywhere or an 'official document' which i can attach to\\ \ 
self-burned cds?]$^{S5}$\vspace{0.1cm}

[I searched the entire web site but found nothing.]$^{S6}$ [Thanks in advance.]$^{S7}$\vspace{0.1cm}

[John]$^{S8}$
                \caption{Source message.}
                \label{fig:exampleSource}
        \end{subfigure}%
}}
\vspace{0.2cm}
\\
       % ~ %add desired spacing between images, e. g. ~, \quad, \qquad etc.
          %(or a blank line to force the subfigure onto a new line)
\fbox {
    \parbox{\linewidth}{
        \begin{subfigure}[b]{0.9\textwidth}
\small
%\footnotesize
%[On Sun, 04 Dec 2005 15:45:13 -0600, John Doe\\
[On Sun, 04 Dec 2005, John Doe 
%On Sun, 05 Dec 2004 16:48:14 -0600, Rich Duzenbury
%<rduz-ubuntu@theduz.com> wrote:
<john@doe.com> wrote:]$^{R1}$\vspace{0.1cm}

> [I got my ubuntu cds today and i'm really impressed.]$^{R2}$ [My\\ \ 
> friends like them and my teachers too (i'm a student).]$^{R3}$\\ \ 
> [It's really funny to see, how people like ubuntu and start feeling geek\\ \ 
> and blaming microsoft when they use it.]$^{R4}$\vspace{0.1cm}

[Rock!]$^{R5}$\vspace{0.1cm}

> [Unfortunately everyone wants an ubuntu cd, so can i download the cd\\ \ 
> covers anywhere or an 'official document' which i can attach to\\ \ 
> self-burned cds?]$^{R6}$\vspace{0.1cm}

[We don't have any for the warty release, but we will have them for hoary, %\\ \ 
because quite a few people have asked. :-)]$^{R7}$\vspace{0.1cm}

[Bob.]$^{R8}$ %\vspace{0.1cm}
%Jon.

%[P.S.]$^{R11}$ [This is a major sticking point for ubuntu and Debian acceptance\\
%on mission critical kit which should be addressed.]$^{R12}$ [It's not too tricky\\
%to boot and initrd off a separate boot partition.]$^{R13}$\vspace{0.1cm}

               % \includegraphics[width=\textwidth]{tiger}
                \caption{Reply message.}
                \label{fig:exampleReply}
        \end{subfigure}
}}
%\end{multicols}

%        ~ %add desired spacing between images, e. g. ~, \quad, \qquad etc.
%          %(or a blank line to force the subfigure onto a new line)
%        \begin{subfigure}[b]{0.4\textwidth}


%\begin{tabular}{*{2}{|l}|c|}
%\toprule
%\textbf{Source} & \textbf{Reply} & \textbf{Label}\\
%	\midrule
%S1  & & \\
%    & R1 & \\
%\textit{S2}  & > \textit{R2}& Start\\
%\textit{S3}  & > \textit{R3}& Inside\\
%\textit{S4}  & > \textit{R4}& End\\
%    & R5 & \\
%    & R6 & \\
%    & R7 & \\
%\textit{S5}  & > \textit{R8} & Start \& End\\
%    & R9 & \\
%%    & R10 & \\
%    & [...] & \\
%S7    &  & \\ \ 
%[...] \    &  & \\
%	\bottomrule
%\end{tabular}
%               % \includegraphics[width=\textwidth]{mouse}
%                \caption{A mouse}
%                \label{fig:mouse}
%        \end{subfigure}
        \caption{Example of a source message and its reply. Pseudo-sentences have been marked. %The original layout has been slightly adapted to fit the document.
        }\label{fig:exampleSourCeReplyMessage}
%\end{multicols}
\end{minipage}
\hfill
\begin{minipage}{.3\textwidth}
\small\centering
\begin{tabular}{*{2}{|l}|c|}
\toprule
\textbf{Source} & \textbf{Reply} & \textbf{Label}\\
	\midrule
S1  & & \\
    & R1 & \\
\textit{S2}  & > \textit{R2}& \texttt{Start}\\
\textit{S3}  & > \textit{R3}& \texttt{Inside}\\
\textit{S4}  & > \textit{R4}& \texttt{End}\\
    & R5 & \\
\textit{S5}  & > \textit{R6} & \texttt{Start\&End}\\
    & R7 & \\
%    & R11 & \\
    & [...] & \\
S6    &  & \\ \ 
[...] \    &  & \\
	\bottomrule
\end{tabular}

\caption{Alignment of sentences from the source and reply messages shown in Figure~\ref{fig:exampleSourCeReplyMessage} and labels inferred from the re-use of source message text. Labels are associated to source sentences.}
\label{fig:exampleSegmentationLabels}
\end{minipage}

\end{figure}




Figure~\ref{fig:exampleSourCeReplyMessage} shows an example of a \textit{source}\footnote{By \textit{source} message, we refer to a message which is replied to. By \textit{original} message, we loosely mean a source message. With this term, we want to point out more precisely the author's contributions.} message (Figure~\ref{fig:exampleSource}) and one of its \textit{reply} (Figure~\ref{fig:exampleReply}).
Sentences have been marked to facilitate the explanations.
In this example, we can see that the reply message only re-uses four selected sentences from the source message; namely $S2$, $S3$, $S4$ and $S5$ which respectively correspond to the sentences  $R2$, $R3$, $R4$ and $R6$ in the reply message.
The author of the reply message deliberately discarded the remaining of the source message.
%
Sentences $S2$, $S3$, $S4$,  and respectively $S5$, can be distinctively associated with two acts: the former segment to a comment, and the latter to a question.


The originality of the proposed approach is to avoid manually annotating the training data and instead exploit the human computational efforts dedicated for a similar task in a different context of production~\cite{ahn:2006:computer}. 
% 
As recommended by the \textit{Netiquette}\footnote{Set of guidelines for Network Etiquette (\textit{Netiquette}) when using network communication or information services \url{http://tools.ietf.org/html/rfc1855}.}, when replying to a message (email or forum post), the writer should 
 ``summarize the original message at the top of its reply, %the message, 
or include (or "quote") just enough text of the original to give a context, in order to make sure readers understand when they start to read the response.''  As a corollary meaning, the writer should ``edit out all the irrelevant material.''
%
% include enough original material to be understood but no more.
Our idea is to use this effort, in particular when the writer replies to a message by inserting his response or comment just after the quoted text appropriate to his intervention. 
%
This posting style is called \textit{interleaved} or \textit{inline replying}.
%L'originalité de l'approche sera de ne pas construire manuellement le corpus d'entraînement mais d'exploiter le temps cognitif investi par d'autres pour une tâche supposée similaire \cite{ahn:2006:computer}. Cela lui vaut le qualificatif de \textit{paresseuse}.
%L'idée est d'exploiter le travail de découpage réalisé par un intervenant lorsqu'il répond à l'intérieur d'un message (\textit{inline replying}).
%
It is true that some email software clients do not conform to the recommendations of Netiquette and that some online participants are less sensitive to arguments about posting style (many writers reply above  the original message).
%In top-posting style, the original message is included verbatim, with the reply above it. 
We assume that there are enough messages with inline replying available to build our training data. 
The so built segmentation model should be usable for any posting styles by applying it only on new content parts.

% Some mail clients, in particular some configurations of Microsoft Outlook, 
%some email software clients are not standards-compliant and 
%some email software clients do not conform to the recommendations of Netiquette.
%Some online participants are less sensitive to arguments about posting style. % (
%Newer online participants, especially those with limited experience of Usenet, tend to be less sensitive to arguments about posting style.

%FIXME When a message is replied to in e-mail, Internet forums, or Usenet, the original can often be included, or "quoted", in a variety of different posting styles.
%
%The main options are {\em interleaved posting} (also called {\em inline replying}, in which the different parts of the reply follow the relevant parts of the original post), \textit{bottom-posting} (in which the reply follows the quote) or \textit{top-posting} (in which the reply precedes the quoted original message). 

We use then a supervised approach to build models of the segmentation. 
We choose to define the segmentation problem as a sequence labelling task whose aim is to assign the globally best set of labels for the entire sequence of sentences at once.


In Section~\ref{sec:buildingannotatedcorpusofsegmentedonlinemessageatnocost}, we explain our approach for building an annotated corpus of segmented online messages at no cost. 
In Section~\ref{sec:buildingTheSegmenter}, we describe the segmentation system we use as well as the features we extract to model the segmentation. 
After presenting our experimental framework in Section~\ref{sec:experimentalframework}, we report some experiments and evaluations of the segmentation system in Section~\ref{sec:experiments}. 
Eventually, we discuss our approach in comparison to other works in Section~\ref{sec:relatedWork}.





%
% The following footnote without marker is needed for the camera-ready
% version of the paper.
% Comment out the instructions (first text) and uncomment the 8 lines
% under final paper for your variant of English.
% 
\blfootnote{
    %
    % for review submission
    %
    \hspace{-0.65cm}  % space normally used by the marker
    % Place licence statement here for the camera-ready version, see
    % Section~\ref{licence} of the instructions for preparing a
    % manuscript.
    %
    % % final paper: en-uk version (to license, a licence)
    %
    % \hspace{-0.65cm}  % space normally used by the marker
    % This work is licensed under a Creative Commons 
    % Attribution 4.0 International Licence.
    % Page numbers and proceedings footer are added by
    % the organisers.
    % Licence details:
    % \url{http://creativecommons.org/licenses/by/4.0/}
    % 
    % % final paper: en-us version (to licence, a license)
    %
    % \hspace{-0.65cm}  % space normally used by the marker
    % This work is licenced under a Creative Commons 
    % Attribution 4.0 International License.
    % Page numbers and proceedings footer are added by
    % the organizers.
    % License details:
    % \url{http://creativecommons.org/licenses/by/4.0/}
}


% -----------------------------------------------------------------------------
% APPROACH

%------------------------------------------------------------------------------
\section{Building annotated corpora of segmented online discussions at no cost}
\label{}

%------------------------------------------------------------------------------
\subsection{Generate the annotations}
\label{}

The basic idea is to take benefit from the act of inserting new content at some specific position in a message initially uttered by a distinct person.
Assumptions in terms of relevance or role in the discourse organisation can be expressed on the new content, on the quoted text or even on the text which is not reused in the reply message.

In this paper, we choose to assume that the sentences of the quoted text in a reply message can inform

The original message consists in an homogeneous discourse flow of utterances. 

By replying to a message and by extracting deliberately some parts\footnote{Summarization operations are also possible.}, the participant performs some cognitive operations leading to identify sufficient information to describe a context.

So the quoted text in a reply message is assumed to be sufficient. 

FIXME develop the idea

The objective is so to determine which parts of the reply messages are reused from the source message.

As declared in the RFC~3676\footnote{\url{http://www.ietf.org/rfc/rfc3676.txt}}, we consider the lines of a message to be quoted if the first character is the quote mark "\texttt{>}".

Unfortunately the process of replying a message lead to some transformations to the original messages. Among them some error of character encoding decoding problem, lines splitting... 

While the form could be acceptable for some basic experiment, we decide to deploy an alignment procedure for various reasons: get the original form of the text which is quoted. Accessing the original message layout for new features. To go further, the original context of the extracted text in the source message also conveys some information.

In the present paper, we have just used the correct form of the 

This is importance since the segmenter is intended to work on non noisy data (new content in message).

The underling idea is to use intrinsic characteristics of the quoted sentences as features for building the future segmentation models. As a consequence noisy data may lead to produce less sure features.


The meaning we give to the term of sentence is more linguistically-based. FIXME



We will assume the greater-than sign to be universal. 




We assume there is no deletion of original text between two consecutive quoted part.

TODO 
\begin{enumerate}
\item 
\item FIXME provide an example of generation and resulting annotation and 
\end{enumerate}

FIXME When a message is replied to in e-mail, Internet forums, or Usenet, the original can often be included, or "quoted", in a variety of different posting styles.
%
The main options are {\em interleaved posting} (also called {\em inline replying}, in which the different parts of the reply follow the relevant parts of the original post), \textit{bottom-posting} (in which the reply follows the quote) or \textit{top-posting} (in which the reply precedes the quoted original message). 


%------------------------------------------------------------------------------
\subsubsection{Annotation scheme}
\label{}


We assume that a message can be split into %subsequent and
 consecutive discourse segments, each of them conveying its own dialogue act.
We assume the sentence as the elementary unit.
Consequently, each sentence in a segment can play one of the following roles: 
\begin{description}
\item [starting and ending] (\textit{SE}) a segment when there is only one sentence in the segment, 
\item [starting] (\textit{S}) a segment if there are at least two sentences in the segment and the sentence is the first one, 
\item [ending] (\textit{E}) a segment if there are at least two sentences in the segment and the sentence is the last one, 
\item [inside] (\textit{I}) a segment in any other cases.
\end{description}
%
As a matter of fact, this scheme is similar to the \textit{BIO} annotation scheme except it is at the sentence level and not at the token level.
% [STARTEND] if the sentence of the source message is surrounded by insertions which are part of the reply message;
%[START] else if the sentence of the source message is surrounded by insertions which are part of the reply message;
%Par exemple, on pourra étiqueter de \texttt{TERMINE} la phrase précédent un segment repris et de \texttt{DEBUTE} la première phrase du segment repris. Ces phrases ainsi annotées dans leur contexte constitueront notre corpus d'entraînement.


We choose to work on the sentences p



%------------------------------------------------------------------------------
\subsubsection{Generation procedure}
\label{}

In the following procedure, the tokens are used to index the quoted lines and the sentences. 




\begin{enumerate}
\item List the source and the reply messages
\item For each pair of source and reply messages
\begin{enumerate}
\item Tokenize both messages in words and sentences
\item Identify % the tokens which are part of 
the quoted lines in the reply message
\item Identify the sentences which are part of the quoted text in the reply message
\item Align the sentences from the source message with the ones from the quoted text in the reply message 
\item For each identified quoted sentences 
\begin{enumerate}
\item Assign a role in terms of segmentation instructions
\end{enumerate}
\end{enumerate}
\end{enumerate}

The identification of a reply message is based on the \texttt{in-reply-to} meta data present in the email headers.



%------------------------------------------------------------------------------
\subsubsection{Alignment module}
\label{}



For finding alignments between two given text messages, we use 
%an implementation of the % Implements a portion of the
% NIST align/scoring 
%algorithm to compare a reference string to a hypothesis string. 
a \textit{dynamic programming (DP) string alignment algorithm} \cite{sankoff:1983}. 
In the context of speech recognition, the algorithm is also known as the \textit{NIST align/scoring algorithm}. Indeed, it is widely used to evaluate the output of speech recognition systems by comparing the hypothesized text (HYP) output by the speech recognizer to the correct, or reference (REF) text. 
In particular, it is used to compute the word error rate (WER) and the sentence error rate (SER).

The ``DP string alignment algorithm performs a global minimization of a Levenshtein distance function which weights the cost of correct words, insertions, deletions and substitutions as 0, 75, 75 and 100 respectively.
%
The computational complexity of DP is $0(NN)$.''
%    final static int MAX_PENALTY = 1000000;
%    final static int SUBSTITUTION_PENALTY = 100;
%    final static int INSERTION_PENALTY = 75;
%    final static int DELETION_PENALTY = 75;
%The alignment and metrics are intended to be, by default, identical to those of the \url{http://www.icsi.berkeley.edu/Speech/docs/sctk-1.2/sclite.htm} NIST SCLITE tool.  
%The program sclite is a tool for scoring and evaluating the output of speech recognition systems. Sclite is part of the NIST SCTK Scoring Tookit. The program compares the hypothesized text (HYP) output by the speech recognizer to the correct, or reference (REF) text. After comparing REF to HYP, (a process called alignment), statistics are gathered during the scoring process and a variety of reports can be produced to summarize the performance of the recognition system.
%\url{http://www1.icsi.berkeley.edu/Speech/docs/sctk-1.2/sclite.htm}



%CMU Sphinx
%Open Source Toolkit For Speech Recognition
%Project by Carnegie Mellon University
% of the performance of a speech recognition or machine translation system.



The Carnegie Mellon University provides an implementation of the algorithm in its speech recognition toolkit\footnote{Sphinx 4 \texttt{edu.cmu.sphinx.util.NISTAlign} %source code.
\url{http://cmusphinx.sourceforge.net}}.
%Implements a portion of the NIST align/scoring algorithm to compare a reference string to a hypothesis string.  It only keeps track of substitutions, insertions, and deletions.
We use an adaptation of it which allows to work on lists of strings\footnote{\url{https://github.com/romanows/WordSequenceAligner}} and not directly on strings (as sequences of characters).


%and other statistics available from an alignment of a hypothesis string and a reference string.

In comparison to the speech recognition and translation use cases, despite some noise, the compared text should include identical parts of the source one.





% -----------------------------------------------------------------------------
\subsection{Building the segmenter}


Our email speech act segmenter system is built around a linear-chain Conditional Random Field (CRF) classifier, as implemented in the sequence labelling toolkit Wapiti \cite{lavergne2010practical}. Each email is processed as a sequence and each sentence as an element of that sequence.

\subsubsection{Features for boundary detection}

The classifier uses features that capture graphic, orthographic, lexical and syntactic information about the sentence. Many of them are borrowed from related work in speech act classification \cite{qadir2011classifying} and email segmentation \cite{lampert2009segmenting}.

\subsubsection{Lexical and syntactic features}

\textbf{Ngrams:} we focus on the first and last three significant tokens in the sentence. Significance is determined by the number of occurrences of the token in the training corpus: terms with a frequency lower than \todo{fix me} 1/X are considered insignificant. 

If a sentence contains less than six significant tokens, the same token can be found in both triplets. For example, in the sentence \textit{``Have a good day !''}, the first three tokens would be \textit{``Have''}, \textit{``a''}, \textit{``good''} and the last three would be \textit{``good''}, \textit{``day''} and \textit{``!''}. If the sentence contains less than three significant tokens, missing values are replaced by a placeholder.

We define three individual features for the three unigrams, the two bigrams and the single trigram found in each of these sets. The features are the following: the unaltered form of the token (case-sensitive), the lemmatized form of the token (case-insensitive ; numbers present in the token are replaced by a special character) and the corresponding part-of-speech.

We use the Stanford Log-linear Part-Of-Speech Tagger for morpho-syntactic tagging \cite{toutanova2003feature}, and the WordNet lexical database to perform lemmatization \cite{miller1995wordnet}.

\textbf{Question words and interrogative ngrams:} one feature checks if a sentence begins with a ``wh*'' question word  (\textit{``who''}, \textit{``when''}, \textit{``where''}, \textit{``what''}, \textit{``which''}, \textit{``what''}, \textit{``how''}) or an ngram suggesting an incoming interrogation (e.g. \textit{``is it''} or \textit{``are there''}), and another one checks if the sentence merely contains such a word or ngram.

\textbf{Modals:} one feature indicates wether the sentence contains a modal (\textit{``may''}, \textit{``must''}, \textit{``shall''}, \textit{``will''}, \textit{``might''}, \textit{``should''}, \textit{``would''}, \textit{``could''}, and their negative forms).

\textbf{Plan phrases:} one feature looks for plan phrases (e.g. \textit{``i will''} or \textit{``we are going to''})

\textbf{Personal words:} three features check for first person, second person and third persons words, respectively (e.g. \textit{``we''}, \textit{``my''} and \textit{``me''} are recognized as first person words).

\textbf{First personal pronoun:} one feature records the first personal pronoun found in the sentence.

\textbf{First verbal form:} one feature records the tag of the first verbal form found in the sentence as classified by the Stanford Part-Of-Speech tagger ; that is an element of the Penn Treebank tag set \footnote{Alphabetical list of part-of-speech tags used in the Penn Treebank Project: \url{http://www.ling.upenn.edu/courses/Fall_2003/ling001/penn_treebank_pos.html}} (e.g. the feature \textit{``VBZ''} indicates a present tense verb in third person singular).

\subsubsection{Graphic features}

The following features' values are relative to other sentences. Four classes are defined for each feature: ``highest'' (top 25\%), ``high'' (top 50\%), ``low'' (bottom 50\%) and ``lowest'' (bottom 25\%).\newline

\textbf{Number of tokens:} the total number of tokens in the sentence, including insignificant ones.

\textbf{Number of characters:} the total number of characters in the sentence, including those in insignificant tokens.

\textbf{Average token length:} the average length of a token, including insignificant ones.

\textbf{Proportion of uppercase characters:} the proportion of uppercase characters in the sentence.

\textbf{Proportion of alphabetic characters:} the proportion of alphabetic characters in the sentence.

\textbf{Proportion of numeric characters:} the proportion of numeric characters in the sentence.

\subsubsection{Orthographic features}

\textbf{Number of greater-than signs:} the number of greater-than signs (``>''), also know as ``chevron'' symbols, in the sentence. Like previous features, this one is computed relatively to other sentences in the training set.

\textbf{Position:} the position of the sentence in the email.

\textbf{Question mark:} one feature checks if the sentence ends with a question mark, and another checks if it at least contains one.

\textbf{Colon:} one feature checks if the sentence ends with a colon, and another checks if it at least contains one.

\textbf{Semicolon:} one feature checks if the sentence ends with a semicolon, and another checks if it at least contains one.

\textbf{Early punctuation:} the last feature checks if the sentence contains any punctuation within the first three tokens of the sentence. This is meant to recognize greetings \cite{qadir2011classifying}.



% -----------------------------------------------------------------------------
% METHODOLOGY


\section{Experimental framework}

\subsection{Corpora}

\subsubsection{Ubuntu technical support email archive}

We use the \textit{ubuntu-users} email archive\footnote{\textit{ubuntu-users} mailing list archive: \url{https://lists.ubuntu.com/archives/ubuntu-users/}} as our primary corpus. It offers a number of advantages. It is free, and distributed under an unrestrictive license. It is up-to-date, with messages as recent as December 2013, and therefore is representative of modern emailing in both content and formatting conventions. Additionally, many alternatives archives are available\footnote{Ubuntu mailing lists archives: \url{https://lists.ubuntu.com/archives/}}, in a number of different languages, including some very resource-poor languages.

\subsubsection{BC3 email corpus}

\todo{fill me}

\subsection{Evaluation protocols}

We designed two evaluation protocols. The first one attempts to assess the segmenter's performance through 10-fold cross-validation on the Ubuntu corpus. The second uses a segmenter trained on the Ubuntu corpus to label the BC3 email corpus, and compares the resulting segmentation to a gold standard built \todo{finish me} from

\subsubsection{Metrics}

Traditional metrics of precision and recall are provided for all results. Precision is the percentage of boundaries identified by the classifier that are indeed true boundaries, recall is the percentage of true boundaries that are identified by the classifier. We also provide the $F_1$ score which represents the harmonic mean of precision and recall:

\[
    F_1 = 2 \cdot \frac{\mathrm{precision} \cdot \mathrm{recall}}{\mathrm{precision} + \mathrm{recall}}
\]

However, automatic evaluation of speech segmentation through these metrics is problematic as predicted segment boundaries rarely align precisely. Moreover, while a segmenter that places boundaries near actual boundary positions is in almost all cases more suited to the task than one that misses by a much larger margin, precision and recall metrics would penalize both to the same extent. Therefore, in order to evaluate varying degrees of success or failure in a more subtle manner, we also provide an array of metrics relevant to the field of text segmentation : $\bm{P_{k}}$, WindowDiff and the Generalized Hamming Distance (GHD).

The $\bm{P_{k}}$ metric is a probabilistically motivated error metric for the assessment of segmentation algorithms \cite{beeferman1999statistical}.

\textbf{WindowDiff} compares the number of segment boundaries found within a fixed-sized window to the number of boundaries found in the same window of text for the reference segmentation \cite{pevzner2002critique}.

The \textbf{Generalized Hamming Distance (GHD)} is an extension of the Hamming distance\footnote{Wikipedia article on the Hamming distance: \url{http://en.wikipedia.org/wiki/Hamming_distance}} to give partial credit for near misses \cite{bookstein2002generalized}.

\subsubsection{Baselines}

Performance is compared to two simple heuristic baselines. The first one, the ``regular'' baseline, is computed by segmenting the test set into regular segments of the same length as the average training set segment length, rounded up. The second one, the ``irregular'' baseline, is computed by randomly inserting an appropriate number of segment boundaries in the test set (i.e. the boundary frequency stays similar to that of the training set). Results provided for the irregular baseline are calculated by computing the average scores of the first 1000 random segmentations.


\subsection{Preprocessing}



% -----------------------------------------------------------------------------
% RESULT

\newcolumntype{C}[1]{>{\centering\let\newline\\\arraybackslash\hspace{0pt}}m{#1}}

\section{Experiments}
\label{sec:experiments}

% FIXME Table \ref{fig:results} présente / résume / fusionne tous les résultats ; phrase qui annonce les différentes étapes

Table\ref{fig:results} shows the summary of all obtained results. On the left side are shown results for segmentation metrics, on the right side results for information retrieval metrics. First, we examine baseline scores, and display them in the top section of Table\ref{fig:results}. Second, in the middle section, we show results for segmenters based on individual feature sets. Finally, in the lower section, we show results based on feature sets combinations. For these experiments the CRF window size is set at 5, i.e. the classification algorithm takes into account features of the next and previous two sentences as well as the current one.

\begin{table}[h]
	\begin{tabular}{p{2.5cm}|C{1.5cm}|C{1.5cm}|C{1.5cm}||C{1.5cm}|C{1.5cm}|C{1.5cm}|}
		\cline{2-7}
		& \multicolumn{3}{c||}{Segmentation metrics} & \multicolumn{3}{c|}{IR metrics} \\ \cline{2-7}
		& \textit{WD} & $P_{k}$ & \textit{GHD} & \textit{Precision} & \textit{Recall} & \textit{$F_1$} \\ \hline
		\multicolumn{1}{|l|}{regular baseline} 					& .59 & .25 & .60 & .31 & .49 & .38  \\ \hline
		\multicolumn{1}{|l|}{TextTiling baseline} 				& .41 & .07 & .38 & .75 & .44 & .56 \\ \hline\hline
		\multicolumn{1}{|l|}{$n$-grams} 						& .38 & \textbf{.05} & .39 & \textbf{1} & .39 & .56 \\ \hline
		\multicolumn{1}{|l|}{information structure} 			& .43 & .11 & .38 & .60 & .68 & \textbf{.64} \\ \hline
		\multicolumn{1}{|l|}{thematic (TextTiling)}				& .39 & .05 & .38 & .94 & .40 & .56 \\ \hline
		\multicolumn{1}{|l|}{miscellaneous} 					& .41 & .09 & .38 & .69 & .49 & .57 \\ \hline\hline
		\multicolumn{1}{|l|}{all features} 						& .38 & \textbf{.05} & .39 & \textbf{1} & .39 & .56\\ \hline
		\multicolumn{1}{|l|}{$n$-grams $\cup$ non-$n$-grams} 	& .45 & .12 & .40 & .58 & \textbf{.69} & .63 \\ \hline
		\multicolumn{1}{|l|}{$n$-grams $\cup$ "cherry-pick"} 	& \textbf{.36} & .06 & \textbf{.34} & .80 & .53 & \textbf{.64} \\ \hline
	\end{tabular}
	\caption{Comparative results for segmenters and baselines. All displayed results show \textit{WindowDiff} (\textit{WD}), $P_{k}$ and \textit{GHD} as error rates, therefore a lower score is desirable for these metrics. This contrasts with the three information retrieval scores, for which a low value denotes poor performance.}
	\label{fig:results}
\end{table}

\subsection{Baseline segmenters}

The first section of Table \ref{fig:results} shows the results obtained by both of our baselines. Unsurprisingly, TextTiling performs much better than the basic regular segmentation algorithm across all metrics.

\subsection{Segmenters based on individual feature sets}

The second section of Table \ref{fig:results} shows the results for four different classifiers, each trained with a distinct subset of the feature set. While all classifiers easily beat the regular baseline, and the TextTiling baseline when it comes to information retrieval metrics, only the thematic and the $n$-grams segmenters manage to surpass TextTiling when performance is measured by segmentation metrics. In terms of IR scores, the $n$-grams classifier in particular stands out as it manage to achieve an outstanding 100\% precision, although this result is mitigated by a meager 39\% recall. It is also interesting to see that the thematic classifier, based only on contextual information about TextTiling output, performs better than the TextTiling baseline.

\subsection{Segmenters based on feature sets combinations}

The last section of Table \ref{fig:results} shows the results of three different segmenters. The first one is simply a classifier that takes all available features into account. Its results are exactly identical to that of the $n$-grams classifiers, most certainly due to the fact that other features are filtered out due to the sheer number of lexical features. The second one segments according to the union of the boundaries detected by a classifier trained on lexical features and those identified by a classifier trained on all other features. Doing this allows the segmenter to obtain the best possible recall, but at the expense of precision. The last one, "$n$-grams $\cup$ cherry-pick"  considers all boundaries detected by a classifier trained on lexical features as correct, and selects only the boundaries identified by a classifier trained on all other features with a confidence score of 99\% or more. This system outperforms all others both in terms of segmentation scores and $F_1$, however it is still relatively conservative and the segmentation ratio (the number of true boundaries divided by the number of guessed boundaries) remain significantly lower than expected, at~0.67.


% -----------------------------------------------------------------------------
% RELATED WORK
%------------------------------------------------------------------------------
\section{Related work}
\label{sec:relatedWork}

Three research areas are directly related to our study:
a) collaborative approaches for acquiring
annotated corpora, b) detection of email structure, and c) sentence alignment.


We can set our approach in the trend of the collaborative approaches for acquiring
annotated corpora such as the Game With A Purpose (GWAP) \cite{ahn:2006:computer} or the paid-for crowdsourcing \cite{fort:2011:cl}.
In the \cite{wang:2013:lre}'s taxonomy, it could be more related to the \textit{Wisdom of the Crowds} (WotC) genre where motivators are altruism or prestige to collaborate for the building of a public resource.
As a major difference, we did not initiate the annotation process and consequently we did not define annotation guidelines, design tasks or develop tools for annotating which are always problematic questions.
We have just rerouted \textit{a posteriori} the result of an existing task which was performed in a distinct context.
In our case the burning issue is to determine the adequacy of our segmentation task.
Our work is motivated by the need to identify important snippets of information in messages for applications such as being able to determine whether all the aspects of a customer request were fully considered.
We argue that even if it is not always obvious to tag topically or rhetorically a segment, the fact that it was a human who actually segmented the message ensures its quality.
%
% is another major genre for crowdsourcing. WotC deployments allow members of the general public to collaborate to build a public resource, to predict event outcomes or to estimate difficulty to guess quantities. Wikipedia, the most well-known WotC instance, has different motivators that have changed over time. Initially, altruism and indirect benefit were factors: people contributed articles to Wikipedia not only to help others but also to build a resource that would ultimately help themselves. As Wikipedia matured, the prestige of being a regular contributor or editor became a motivator (Suh et al. 2009).
%
We think that our approach can also be used for determining the relevance of the segments, however it has some limits, and we do not know how labeling segments with dialogue acts may help us do so.

Detecting the structure of a thread is a hot topic. 
%
As mentioned in Section~\ref{sec:intro}, very little works have been done on email segmentation. 
We are aware of recent works in linear text segmentation such as \cite{kazantseva:2011} who address the problem by modelling the text as a graph of sentences and by performing clustering and/or cut methods. 
%
Due to the size of the messages (and consequently the available lexical material), it is not always possible to exploit this kind of method. However, our results tend to indicate that we should investigate in this direction nonetheless.
%
By detecting sub-units of information within the message, our work may complement the works of \cite{li:2011:threadlinking,kim:2010:taggingandlinking} who proposes solutions for detecting links between messages. 
% \texttt{in-reply-to} link
We may extend these approaches by considering the possibility of pointing to multiple message targets or determining more precisely the pointed area.

Concerning the alignment process, our task can be compared to the detection of monolingual text derivation (otherwise called plagiarism, near–duplication, revision). \cite{poulard:2011:detecting} compare, for instance, the use of $n$–grams overlap with the use of text specificities. 
In comparison, in our work we already know that a text (the reply message) derives from another (the source message). Sentence alignment has also been a very active field of research 
%both in monolingual (e.g. plagiarism detection) and multilingual  (e.g. 
in statistical machine translation for building parallel corpora. %) domains. 
%
%In MT, 
Some methods are based on sentence length comparison \cite{gale:1991}, some methods rely on the overlap of rare words (cognates and named entities) \cite{enright-kondrak:2007:ShortPapers}.
%For detection of derivation links, \cite{poulard:2011:detecting} compare the use of n–grams overlap with the use of text specificities. % exploitation of the specificity and invariance of textual elements. 
% between texts (otherwise called plagiarism, near–duplication, revision, etc.) at the document level. 
%We evaluate the use of textual elements implementing the ideas of specificity and invariance as well as their combination to characterize derivatives. We built a French press corpus based on Wikinews 
% revisions to run this evaluation. We obtain performances similar to the state of the art method 
% (n–grams overlap) while reducing the signature size and so, the processing costs. In order ...
In comparison, %to the speech recognition and translation use cases, 
%our work is more an alignment task than a detection of derivation. In addition 
in our task, despite some noise, the compared text includes large parts of material identical to the source text. 
The kinds of edit operation in presence (no inversion\footnote{When computing the Levenshtein distance, the inversion edit operation is the most costly operation.} only deletion, insertion and substitution) lead us to consider the Levenshtein distance as a serious option.  

% \url{http://www.statmt.org/survey/Topic/SentenceAlignment}
% An influential early method is based on sentence length, measured in words (Brown et al., 1991; Gale and Church, 1991; Gale and Church, 1993) or characters
% training models with parallel texts
%Enright and Kondrak (2007) use a simple and fast method for document alignment that relies of overlap of rare but identically spelled words, which are mostly cognates, names, and numbers.



% -----------------------------------------------------------------------------
% FUTURE WORK


%------------------------------------------------------------------------------
\section{Future work}
\label{sec:futureWork}

The main contribution of this work is to exploit the human effort dedicated to reply formatting for training discursive email segmenters. 
We have implemented and tested various segmenter models. 
There is still room for improvement, but our results indicate that the approach merits more thorough examination.
%
Our segmentation approach remains relatively simple and can be easily extended. One way would be to consider contextual features in order to characterize the sentences in the source message structure.
%
As future works, we plan to complete our current experiments with two new approaches for evaluation. The first one will consists in comparing the automatic segmentation with those performed by human annotators.
This task remains tedious since it will then be necessary to define an annotation protocol, write guidelines and build other resources.
The second evaluation we plan to perform is an extrinsic evaluation. The idea will be to measure the contribution of the segmentation in the process of detecting the dialogue acts, i.e. to check if existing sentence-level classification systems would perform better with such contextual information. % at segment-level.

%TODO
%\begin{itemize}
%\item NH finish the state-of-the-art, find a new example and handle/describe the example and our idea
%\item baseline start at each new sentence
%\item segmentation de choi as feature
%\item feature n-grams : only most frequent unigrams
%\item feature as individual sets: graphic+orthographic~stylistic, semantic
%\item feature contextual: empty lines before/after ; line position, lexical similarity before/after sentence
%\item combination of the individual decision of the segmenters
%\item alternatives of ubuntus allows us to build models for many languages at low cost
%\item alignements et étiquetage sur les mots
%\item évaluation par rapport à annotation manuelle
%\item évaluation de apport par rapport à reconnaissance automatique des DA 
%\item un exemple de segmentation que l'on souhaite
%\item aligner seulement les quoted lines et non tout le contenu du reply
%\item travailler la tokenization des emails
%\item Focus on the first message of a thread and its reply messages to avoid noisy data due to several levels of quoted lines.
%\item Restrict on short messages as an heuristic to filter out the long code trace message
%\item Due to alignment complexity focus only on the first 25 000 tokens.
%\item In source message, focus on part which is new ; in reply message, focus on part which is new + only the first? quoted level 
%\item The classifiers are based on common features for discourse analysis \cite{joty:2013:acl}.
%\item why processing thread is difficult: Futhermore, because of some user preferences or systems configurations, a thread may embed and interleave various posting styles.  ...
 
%\end{itemize}



% include your own bib file like this:
\bibliographystyle{acl}
\bibliography{refs}

\end{document}
