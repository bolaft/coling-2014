
%------------------------------------------------------------------------------
\section{Introduction}
\label{sec:intro}

Automatic processing of online conversations (forum, emails) about information or assistance needs 
is a highly important issue for the industrial and the scientific communities who care to improve the present question/answering systems, identify the emotion or the intention in customer requests or reviews, detect messages containing requests for action, unsolved severe problems\ldots

%L'analyse automatique des conversations écrites en ligne asynchrones (e.g. forum, courriel) portant des demandes d'information (e.g. dépannage et assistance) constituent un fort enjeu scientifique et industriel : amélioration des systèmes de question/réponse actuels, détection de revues négatives de produits par des clients, identification de problèmes sévères non-résolus... 



%2010_gupta_W10-0202_email-Emotion-Detection-in-Email-Customer-Care
%2012_vinodkumar_email_Annotations-for-Power-Relations-on-Email-Threads
%2010_kim-W10-0507_modeling-social-and-content-dynamics-in-discussion-forum
%2012_qu_E12-1037_recommendation-prediction-of-user-interest-in-forum
%2013_chen_N13-1124_Identifying-Intention-Posts-in-Discussion-Forums
%2011_marin_W11-0706_detecting-forum-authority-claims
%2010_wang_P10-1027_recommendation-in-forum-blog
%2009_Stavrianou_asonam_Definition and Measures of an Opinion Model for Mining Forums
%2009_arora_N09-2010_Identifying Types of Claims in Online Customer Reviews.pdf


%2013_ott_Negative Deceptive Opinion Spam

%2011_qu_I11-1164_finding-problem-solving-threads-in-forum
%2010_lampert_naacl_Detecting Emails Containing Requests for Action


%Les travaux existants se fondent généralement sur une pré-annotation des messages en actes du dialogue (\textit{DA}). Les \textit{DA} décrivent les fonctions communicatives portées dans chaque message (e.g. question, réponse, remerciement...) Austin FIXME. Kim FIXME proposent une liste d'actes spécifiques à la description des messages au sein de forums.
%Jusqu'à présent la majorité des travaux ont modélisé les discussions au niveau de leurs messages en les décrivant su

In most of the works, conversation interactions between the participants are modelled in terms of the dialogue act (DA) \cite{austin:1970}. The DA describe the communicative function conveyed by each text utterance  (e.g.~question, answer, greeting,\ldots).
%La majorité des travaux modélisent les interactions entre intervenants en termes d'\textit{actes de dialogue} (DA) \cite{austin:1970}. Les DAs correspondent aux fonctions communicatives portées par chaque énoncé d'un texte (e.g. question, réponse, remerciement...). 
% La théorie des actes du dialogue \cite{austin:1970} propose de décrire les énoncés en termes des fonctions communicatives portées par chacun d'eux (e.g. question, réponse, remerciement...). 
%
The main trend in automatic DA recognition consists in using supervised learning algorithms to predict the DA conveyed by a sentence or a message \cite{joty:2013:sigdial}.
% L'approche dominante en reconnaissance automatique de DA consiste en l'usage d'algorithmes de classification supervisée \cite{joty:2013:sigdial} pour déterminer l'acte porté par une phrase ou un message. 
%
The segmentation of the messages results from the analysis of these individual predictions.
%Le découpage des messages découle alors des résultats d'analyse.
%
A first remark of this paradigm is that it is not realistic to use it in the context of multi-domain and multimodal processing because it requires the building of training data which is a very substantial and time-consuming task.
%Une première critique de ce paradigme est que sa mise en oeuvre est laborieuse et coûteuse en temps d'annotation pour construire des données d'entraînement, ce qui n'est pas réaliste en contexte de traitement multi-domaine voire multi-modal.

A second remark is that the model does not have a fine-grained representation of the message structure or the relations between the messages. Considering such characteristics could drastically improve the systems for example to focus on specific text parts or to filter out the less relevant ones. 
% Une seconde critique est que dans ce contexte, la connaissance de la structure des messages permettrait aux applications susvisées de se focaliser sur des parties spécifiques et filtrer les passages moins pertinents. 
Indeed, apart from the closing formula, a message may be made of several distinct information requests, the description of unsuccessful procedure, the quote of third-party messages\ldots
% En effet, outre des formules de politesse, un message peut compter par exemple plusieurs expressions de besoin, décrire une procédure infructueuse et citer des portions de messages tiers. 
% faire référence à plusieurs contenus exprimés par différentes contributions tierces
So far, a few of works address the problem of email segmentation.
\cite{lampert:2009:emnlp} propose to segment emails in prototypical zones such as the author's contribution, the quotes of original messages, the signature, the opening and the closing formulas. 
\cite{joty:2013:jair} identifies topical segments. 
%De par sa robustesse, cette dernière approche peut servir de base de comparaison.
%Or, peu de travaux se sont penchés sur la tâche de segmentation des courriels en tant que telle. Les rares travaux ne concernent pas le découpage en DA. \cite{lampert:2009:emnlp} s'intéressent à découper les courriels en des zones prototypiques telles que la contribution de l'auteur, les reprises de messages tiers, la signature et les formules d'appel et de clôture... \cite{joty:2013:jair} identifient des segments thématiques. De par sa robustesse, cette dernière approche peut servir de base de comparaison.

This paper addresses the problem of rhetorically segmenting the new content parts of messages in online discussions. The process aims at supporting the analysis of messages in terms of dialogues acts.
%Ce sujet propose de travailler sur la tâche de segmentation des courriels en français dans des discussions portant sur des demandes d'information. Ce travail vise à soutenir une analyse automatique des messages en DA.

Despite the drawbacks mentioned above, a supervised approach remains generally the most efficient and reliable method to solve classification problems in Natural Language Processing. 
% 
Our objective is to train a system to detect the segment boundaries; In other words to classify if a given sentence starts, ends or continues a segment.

FIXME give an example of expected segmentation by referring to the Figure 

The originality of the proposed approach is not to manually annotate the training data but to exploit the human computational efforts dedicated for a similar task in a different context of production~\cite{ahn:2006:computer}. 
% 
As recommended by the \textit{Netiquette}\footnote{Set of guidelines for Network Etiquette (\textit{Netiquette}) when using network communication or information services \url{http://tools.ietf.org/html/rfc1855}.}, when replying to a message (email or forum post), the writer should 
 ``summarize the original message at the top of its reply, %the message, 
or include (or "quote") just enough text of the original to give a context, in order to make sure readers understand when they start to read the response.''  As a corollary meaning, the writer should ``edit out all the irrelevant material.''
%
% include enough original material to be understood but no more.
Our idea is to use this effort, in particular when the writer replies to a message by inserting his response or comment just after the quoted text appropriate to his intervention. 
%
This posting style is called \textit{interleaved} or \textit{inline replying}.
%L'originalité de l'approche sera de ne pas construire manuellement le corpus d'entraînement mais d'exploiter le temps cognitif investi par d'autres pour une tâche supposée similaire \cite{ahn:2006:computer}. Cela lui vaut le qualificatif de \textit{paresseuse}.
%L'idée est d'exploiter le travail de découpage réalisé par un intervenant lorsqu'il répond à l'intérieur d'un message (\textit{inline replying}).
%
It is true that some email software clients do not conform to the recommendations of Netiquette and that some online participants are less sensitive to arguments about posting style (many writers reply above  the original message).
%In top-posting style, the original message is included verbatim, with the reply above it. 
We assume that there are enough messages with inline replying available to build our training data. 
The so built segmentation model be usable for any posting styles by applying it only on new content parts.

% Some mail clients, in particular some configurations of Microsoft Outlook, 
%some email software clients are not standards-compliant and 
%some email software clients do not conform to the recommendations of Netiquette.
%Some online participants are less sensitive to arguments about posting style. % (
%Newer online participants, especially those with limited experience of Usenet, tend to be less sensitive to arguments about posting style.

%FIXME When a message is replied to in e-mail, Internet forums, or Usenet, the original can often be included, or "quoted", in a variety of different posting styles.
%
%The main options are {\em interleaved posting} (also called {\em inline replying}, in which the different parts of the reply follow the relevant parts of the original post), \textit{bottom-posting} (in which the reply follows the quote) or \textit{top-posting} (in which the reply precedes the quoted original message). 

We use then a supervised approach to build models of the segmentation. The classifiers are based on common features for discourse analysis \cite{joty:2013:acl}.



