
\newcolumntype{C}[1]{>{\centering\let\newline\\\arraybackslash\hspace{0pt}}m{#1}}

\section{Experiments}
\label{sec:experiments}

FIXME Table \ref{fig:results} présente / résume / fusionne tous les résultats ; phrase qui annonce les différentes étapes

\begin{table}[h]
	\begin{tabular}{p{1.5cm}|C{1.5cm}|C{1.5cm}|C{1.5cm}||C{1.5cm}|C{1.5cm}|C{1.5cm}|}
		\cline{2-7}
		& \multicolumn{3}{c||}{Segmentation metrics} & \multicolumn{3}{c|}{IR metrics} \\ \cline{2-7}
		& \textit{WD} & $P_{k}$ & \textit{GHD} & \textit{Precision} & \textit{Recall} & \textit{$F_1$} \\ \hline
		\multicolumn{1}{|l|}{regular baseline} 					&  &  &  &  &  &  \\ \hline
		\multicolumn{1}{|l|}{TextTiling baseline} 				&  &  &  &  &  &  \\ \hline\hline
		\multicolumn{1}{|l|}{$n$-grams} 						&  &  &  &  &  &  \\ \hline
		\multicolumn{1}{|l|}{information structure} 			&  &  &  &  &  &  \\ \hline
		\multicolumn{1}{|l|}{thematic (TextTiling)}				&  &  &  &  &  &  \\ \hline
		\multicolumn{1}{|l|}{miscellaneous} 					&  &  &  &  &  &  \\ \hline\hline
		\multicolumn{1}{|l|}{all features} 						&  &  &  &  &  &  \\ \hline
		\multicolumn{1}{|l|}{$n$-grams $\cup$ non-$n$-grams} 	&  &  &  &  &  &  \\ \hline
		\multicolumn{1}{|l|}{$n$-grams $\cup$ "cherry-pick"} 	&  &  &  &  &  &  \\ \hline
	\end{tabular}
	\caption{Comparative results for segmenters and baselines. All displayed results show \textit{WindowDiff} (\textit{WD}), $P_{k}$ and \textit{GHD} as error rates, therefore a lower score is desirable for these metrics. This contrasts with the three information retrieval scores, for which a low percentage denotes poor performance.}
	\label{fig:results}
\end{table}

\subsection{Baseline segmenters}

The first section of Table \ref{fig:results} shows the results obtained by both of our baselines. Unsurprisingly, TextTiling performs much better than the basic regular segmentation algorithm across all metrics.

\subsection{Segmenters based on individual feature sets}

The second section of Table \ref{fig:results} shows the results for four different classifiers, each trained with a distinct subset of the feature set. While all classifiers easily beat the regular baseline, none manage to surpass TextTiling when performance is measured by segmentation metrics. Some results however stand out. In particular, the classifier based on lexical features manage to achieve an outstanding TODO\% precision, although this result is mitigated by a meager TODO\% recall.

\subsection{Segmenters based on feature set combination}

The last section of Table \ref{fig:results} shows the results of three different segmenters. The first one is simply a classifier that takes all available features into account. The second one segments according to the union of the boundaries detected by a classifier trained on lexical features and those identified by a classifier trained on all other features. The last one considers all boundaries detected by a classifier trained on lexical features, and selects some boundaries identified by a classifier trained on all other features according to their confidence score. The number of "cherry-picked" boundaries depends on the expected number of boundaries in the test set, which is computed from the average segment length in the training set.