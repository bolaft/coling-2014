%------------------------------------------------------------------------------
\section{Related work}
\label{sec:relatedWork}

Three research areas are directly related to our study:
a) text segmentation models, b) probabilistic topic
models, and c) extracting and representing the con-
versation structure of emails.

Again, our problem of
topic segmentation for emails is not sequential in na-
ture.

We can set our approach in the trend of the collaborative approaches for acquiring
annotated corpora such as the Game With A Purpose (GWAP) \cite{ahn:2006:computer} or the paid-for crowdsourcing \cite{fort:2011:cl}.
In the \cite{wang:2013:lre}'s taxonomy, it could be more related to the \textit{Wisdom of the Crowds} (WotC) genre where motivators are altruism or prestige to collaborate for the building of a public resource.
As a major difference, we did not initiate the annotation process and consequently we did not define any annotation guidelines, design tasks or develop tools for annotating which are always problematic questions.
We have just reroute \textit{a posteriori} the result of an existing task which was performed in a distinct context.
In our case the burning issue is so to determine the adequacy to our segmentation task.
Our work is motivated by the need of identifying important snippets of information in messages for applications such as being able to determine whether all the aspects of a customer request were fully considered.
We argue that even if it is not always obvious to tag topically or rhetorically a segment, the fact that it was a human who actually segmented the message ensures its quality.
%
% is another major genre for crowdsourcing. WotC deployments allow members of the general public to collaborate to build a public resource, to predict event outcomes or to estimate difficulty to guess quantities. Wikipedia, the most well-known WotC instance, has different motivators that have changed over time. Initially, altruism and indirect benefit were factors: people contributed articles to Wikipedia not only to help others but also to build a resource that would ultimately help themselves. As Wikipedia matured, the prestige of being a regular contributor or editor became a motivator (Suh et al. 2009).
%
We think that our approach can also be used for determining the relevance of the segments. But it has some limits, we do not know how it may help us to label the segments with dialogue acts.

Detecting the structure of a thread is a hot topic. Our work may complement the works of \cite{li:2011:threadlinking,kim:2010:taggingandlinking} who proposes solutions for detecting links between messages. 
% \texttt{in-reply-to} link
By detecting sub-units of information within the message, we may extend these approaches by considering the possibility of pointing to multiple message targets.

Concerning the alignment process, sentence alignment has been a very active field of research of statistical machine translation. 
Some methods are based on sentence length comparison \cite{gale:1991}, some methods rely on the overlap of rare words (cognates and named entities) \cite{enright-kondrak:2007:ShortPapers}.



% \url{http://www.statmt.org/survey/Topic/SentenceAlignment}
% An influential early method is based on sentence length, measured in words (Brown et al., 1991; Gale and Church, 1991; Gale and Church, 1993) or characters
% training models with parallel texts
%Enright and Kondrak (2007) use a simple and fast method for document alignment that relies of overlap of rare but identically spelled words, which are mostly cognates, names, and numbers.

\begin{itemize}
\item discussion en ligne (annotation en DA, reconnaissance de la structure)
\item  techniques d'alignement : monolingual text alignment, détection de plagiat, leviestein intéressante car dans notre cas pas de cas d'inversion ce qui est une des opérations couteuses à appréhender avec leviestein lorsqu'elle est considérée.


\item  techniques de segmentation
\item In comparison to the speech recognition and translation use cases, despite some noise, the compared text should include identical parts of the source one.

\end{itemize}
